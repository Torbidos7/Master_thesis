\documentclass[../main.tex]{subfiles}
\usepackage[english]{babel}
\graphicspath{{\subfix{../images}}}
\begin{document}
In this work we proposed the application of an ASSL training strategy in combination with TL for the automated segmentation of pet wound images. 
Although we classify the entire set of data as natural images, with uncontrolled illumination, background, and exposition, the usage of natural light reflected images showed robust results for the segmentation of wound areas.
However some limits arise when considering the effective calculation of dimensions which could be performed only with the use of markers as reference points.
The introduction of a standardized acquisition protocol will be helpful to exploits all the information contained in this kind of images.

With the proposed pipeline, we confirmed the effectiveness of ASSL in labelling large amount of data with a minimal human effort. 
The introduction of TL in the training pipeline allows to start the ASSL without an initial manually annotated dataset, further reducing the workload required by the clinicians. 
In the ASSL procedure, the experts evaluation consists of an accept/discard method, minimizing the time and the effort required for the generation of high-quality segmentation masks.

\subsection{Deepskin}

The ASSL implementation for EfficientNet model trained on Deepskin produced remarkable results: starting from a relatively small core set of manually annotated samples,  we have been able to obtain annotations for more than 90\% of the available images in just four rounds of training.

These segmentations were used as ground truth for the training of MobileNet model, which achieved IoU and $F_1$ scores of 0.89 and 0.94 on the validation set, respectively, after 100 epochs of training, showing robust and comparable results with respect to EfficientNet ones.

The quality of the results on different types of wounds is showed in Figure \ref{fig:deep-visual-results}: even if we can spot some differences in the borders, the wound area is well defined with both models and there are no spots inside the selected region.
The results on Deepskin represent a reliable starting point for the TL on Petwound.

\subsection{Petwound}

TL technique was applied to perform the Round 0 segmentation on Petwound for both models. 
This application reached a total of 143 correctly annotated images with EfficientNet and 113 images with MobileNet, corresponding to 49\% and 39\% of images belonging to Petwound dataset. 
These results remark the effectiveness of TL and confirm the assumptions on the relation between the two datasets in the feature space. 
The application of TL allowed the models to be trained on Petwound dataset without any annotated images at all.
This result could represent a fundamental baseline for future model development, when considering two related datasets.

The EfficientNet model after five rounds of training achieved an $F_1$ score of 0.98 and an IoU score of 0.96 on the validation set, confirming the high quality of the generated segmentations.
As showed in Table \ref{tab:results-eff-petwound}, the model performances are consistent through the rounds of ASSL training with a slight improvement of the two considered metrics with respect to the values of the initial round. 
We must consider the model performance as positively biased because, at each round, we were including only images for which we knew that the model performed well in the previous round.
However, the positive trend of the correctly segmented images along the ASSL rounds shows that the model is improving its prediction and generalization capabilities. 
The robustness of our training pipeline is enforced by the fact that the EfficientNet U-Net model, trained on the Deepskin dataset, generalized also on the PetWound dataset, correctly annotating almost 50\% of the available images without any fine-tuning of the model.

The results obtained during the ASSL using the MobileNet U-Net model were not as good as the ones obtained with the EfficientNet.
We argue that the MobileNet backbone could be not powerful enough to perform this complex segmentation task in the initial stages of ASSL, when the number of available samples is extremely limited. 
After the training on the annotation produced by EfficientNet, the performance improvement of MobileNet model suggests that a simpler model as the MobileNet one could need a greater number of training samples to learn the meaningful characteristics of the images in complex contests as the one of this study.
This result goes against the classical belief that optimal number of parameters, involved in a deep learning model, scales with the quantity of available data: the introduction of TL in the pipeline seems to remedy to this problem, providing a sufficiently good starting point for the correct generalization on new data.

To our knowledge, there are no available segmentation datasets of animal wound images, thus, the PetWound images collected, and the corresponding segmentations produced in this work could be used as a valid benchmark for the application of deep learning pipelines on similar studies. An ASSL strategy having the PetWound dataset as starting point could produce easily hundreds of other animal wound annotated images.
The main limitation of our dataset is the restricted number of available images and the inclusion of only two different pet species, namely dogs and cats. The expansion of the dataset with wound images of other animal species would be beneficial for the generalizability of the study. On the other hand, a strength of our dataset is the high heterogeneity of the included images, presenting different resolutions, light conditions, and backgrounds, making the dataset suitable for real clinical practice scenarios.
\\
\\

Considering the total results on both datasets and the benefits of the ASSL procedure, we can assert that this procedure could drastically improve the availability of huge, annotated datasets with a minimum time-cost required by expert clinicians. 
In the proposed procedure, the experts evaluation consists of an accept/discard method, while the actual segmentation is performed by the model predictions.
With the ASSL procedure the production of hundreds of viable pixel-wise segmentations could be performed in few hours of evaluation; whereas, with manual segmentation, clinicians could spend up to couple of hours for a single wound image among the most difficult ones.
Moreover, with the addiction of TL training strategies, experts do not have to produce manual segmentation for any of the images on the Petwound dataset, notably reducing the effort needed. 
\end{document}